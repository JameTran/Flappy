\documentclass{article}

\usepackage{tabularx}
\usepackage{booktabs}

\title{SE 3XA3: Problem Statement\\Untitled}

\author{Team \#4, Undetermined
		\\ Andrew Hum | huma3
		\\ William Lei | leim5
		\\ Arshan Khan | khana172
		\\ Jame Tran | tranj52
}

\date{January 24, 2019}

% \input{../Comments} %Cannot find a comments file in the parent directory

\begin{document}

\begin{table}[hp]
\caption{Revision History} \label{TblRevisionHistory}
\begin{tabularx}{\textwidth}{llX}
\toprule
\textbf{Date} & \textbf{Developer(s)} & \textbf{Change}\\
\midrule
1/21/2020 & Andrew Hum & Format document and begin initial draft.\\
Date2 & Name(s) & Description of changes\\
... & ... & ...\\
\bottomrule
\end{tabularx}
\end{table}

\newpage

\maketitle

Redesign simple Python games to increase the graphics and complexity.
Alongside the redesign, there will be a new launcher.

- A clear concise description of the issues that need to be addressed by your team
- What problem are you trying to solve? Not how you are going to solve the problem

Currently, we have a collection of very simple minigames that can be played by launching each game individually. These games have 
low graphics quality, minimal features and poor performance. To play a game, you have to launch them using a Linux subsystem, and each
game must be launched individually.

- Why is this an important problem?

It is a univeral fact that games should be entertaining, with smooth performance. We want these simple minigames to provide said
entertainment to the users, as well as ease of access.

- What is the context of the problem you are solving?
	- Stakeholders?
	- Software environment?

%\wss{comment}

%\ds{comment}

%\mj{comment}

%\cm{comment}

%\mh{comment}

\end{document}