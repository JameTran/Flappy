\documentclass[12pt, titlepage]{article}

\usepackage{fullpage}
\usepackage[round]{natbib}
\usepackage{multirow}
\usepackage{booktabs}
\usepackage{tabularx}
\usepackage{graphicx}
\usepackage{float}
\usepackage{placeins}
\usepackage{hyperref}
\hypersetup{
    colorlinks,
    citecolor=black,
    filecolor=black,
    linkcolor=red,
    urlcolor=blue
}
\usepackage[round]{natbib}

\newcounter{acnum}
\newcommand{\actheacnum}{AC\theacnum}
\newcommand{\acref}[1]{AC\ref{#1}}

\newcounter{ucnum}
\newcommand{\uctheucnum}{UC\theucnum}
\newcommand{\uref}[1]{UC\ref{#1}}

\newcounter{mnum}
\newcommand{\mthemnum}{M\themnum}
\newcommand{\mref}[1]{M\ref{#1}}

\title{SE 3XA3: Test Report\\Mini-Arcade}

\author{Andrew Hum \\ huma3 \\ 400138826 \and
		William Lei \\ leim5 \\ 400125240 \and
		Arshan Khan \\ khana172 \\ 400145605 \and
		Jame Tran \\ tranj52 \\ 400144141
}

\date{}


\begin{document}
\maketitle
\newpage
\tableofcontents
\listoftables
\listoffigures

\begin{table}[!hbp]
\caption{\bf Revision History}
\begin{tabularx}{\textwidth}{p{3cm}p{2cm}X}
\toprule {\bf Date} & {\bf Version} & {\bf Notes} \\
\midrule
4/2/2020 & 1.0 & Andrew - Formatted Document \& Section 1 \\
\bottomrule
\end{tabularx}
\end{table}

\newpage

\section{Functional Qualities Evaluation}

\textbf{Description of Tests}: These tests are used to demonstrate that the requirements of the product are fulfilled based upon the functional requirements from the Software Requirements Specification. These tests will demonstrate basic functionality of the game and its correctness. \\
	
	
	Test Name: FRCG
	
	Results: The user is able to choose a game to play \\
	
	Test Name: FRSS
	
	Results: The user is able to access and see the scoreboard \\ 
	
	Test Name: FRPM 
	
	Results: The user is able to play the Maze game \\ 
	
	Test Name: FRPP 
	
	Results: The user is able to play the Pong Game \\ 
	
	Test Name: FRPF
	
	Results: The user is able to play the Flappy Game \\
	
	Test Name: FRRL
	
	Results: The user is able to return to the launcher \\
	
	Test Name: FRME 
	
	Result: The user is able access all of the menu elements for each game \\
	
	Test Name: FRKM
	
	Result: The user is able to move the player using the keyboard or mouse (varies per game) \\
	
	Test Name: FRSG
	
	Result: The game produces a final score and records it for the scoreboard \\
	

\section{Nonfunctional Qualities Evaluation} 

	\subsection{Usability}
	
	Description of Tests: \\
	\\
	Test Name:\\
	Results: \\
	\\	
	
	\subsubsection{Example Test Scenario}
	
	Description of Tests: \\
	\\
	Test Name:\\
	Results: \\
	\\		
	\\	

	\subsection{Performance and Robustness}
	
	\subsubsection{Example Test Scenario}
	
	Description of Tests: \\
	\\
	Test Name:\\
	Results: \\
	\\	

\section{Changes Due to Testing}

	\subsection{Launcher Testing}

	\subsection{Scoreboard Testing}

	\subsection{Maze Testing}
	
	\subsection{Pong Testing}
	
	\subsection{Flappy Testing}

\section{Automated Testing}	

	\subsection{Launcher Testing}
	
	\subsection{Scoreboard Testing}
		
	\subsection{Maze Testing}
		Description of tests: \\
		
		Results: \\ \\
		
	\subsection{Pong Testing}
		Description of tests: \\
		
		Results: \\ \\
		
	\subsection{Flappy Testing}
		Description of tests: \\
		
		Results: \\ \\
		
\section{System Tests}

\subsection{Launcher Testing}

\textbf{Test Name} - \\
\textbf{Initial State} - \\
\textbf{Input} - \\
\textbf{Expected Output} - \\

\subsection{Scoreboard Testing}

\textbf{Test Name} - \\
\textbf{Initial State} - \\
\textbf{Input} - \\
\textbf{Expected Output} - \\

\subsection{Maze Testing}

\textbf{Test Name} - \\
\textbf{Initial State} - \\
\textbf{Input} - \\
\textbf{Expected Output} - \\

\subsection{Pong Testing}

\textbf{Test Name} - \\
\textbf{Initial State} - \\
\textbf{Input} - \\
\textbf{Expected Output} - \\

\subsection{Flappy Testing}

\textbf{Test Name} - \\
\textbf{Initial State} - \\
\textbf{Input} - \\
\textbf{Expected Output} - \\

\section{Trace to Requirements}	

\subsection{Functional Requirements Traceability Matrix}
\begin{table}[H]
\centering
\begin{tabular}{p{0.2\textwidth} p{0.6\textwidth}}
\toprule
\textbf{Test} & \textbf{Req.}\\
\midrule
FT-1 & -\\
FT-2 & -\\
FT-3 & -\\
FT-4 & -\\
FT-5 & -\\
FT-6 & -\\
FT-7 & -\\
FT-8 & -\\
FT-9 & -\\
\bottomrule
\end{tabular}
\caption{Trace Between Tests and Functional Requirements}
\label{TblTFR}
\end{table}

\newpage

\subsection{Non-Functional Requirements Traceability Matrix}
\begin{table}[H]
\centering
\begin{tabular}{p{0.2\textwidth} p{0.6\textwidth}}
\toprule
\textbf{Test} & \textbf{Req.}\\
\midrule
NFT-1 & -\\
NFT-2 & -\\
NFT-3 & -\\
NFT-4 & -\\
NFT-5 & -\\
NFT-6 & -\\
NFT-7 & -\\
NFT-8 & -\\
NFT-9 & -\\
\bottomrule
\end{tabular}
\caption{Trace Between Tests and Non-Functional Requirements}
\label{TblTNFR}
\end{table}

\newpage

\subsection{Automated Testing Traceability Matrix}
\begin{table}[H]
\centering
\begin{tabular}{p{0.2\textwidth} p{0.6\textwidth}}
\toprule
\textbf{Test} & \textbf{Req.}\\
\midrule
AT-1 & -\\
AT-2 & -\\
AT-3 & -\\
AT-4 & -\\
AT-5 & -\\
AT-6 & -\\
AT-7 & -\\
AT-8 & -\\
AT-9 & -\\
\bottomrule
\end{tabular}
\caption{Trace Between Automated Tests and Requirements}
\label{TblTATR}
\end{table}

\newpage

\section{Trace to Modules}

\subsection{Functional Requirements Traceability Matrix}
\begin{table}[H]
\centering
\begin{tabular}{p{0.2\textwidth} p{0.6\textwidth}}
\toprule
\textbf{Test} & \textbf{Modules}\\
\midrule
FT-1 & -\\
FT-2 & -\\
FT-3 & -\\
FT-4 & -\\
FT-5 & -\\
FT-6 & -\\
FT-7 & -\\
FT-8 & -\\
FT-9 & -\\
\bottomrule
\end{tabular}
\caption{Trace Between Functional Tests and Modules}
\label{TblFTM}
\end{table}

\newpage

\subsection{Non-Functional Requirements Traceability Matrix}
\begin{table}[H]
\centering
\begin{tabular}{p{0.2\textwidth} p{0.6\textwidth}}
\toprule
\textbf{Test} & \textbf{Modules}\\
\midrule
NFT-1 & -\\
NFT-2 & -\\
NFT-3 & -\\
NFT-4 & -\\
NFT-5 & -\\
NFT-6 & -\\
NFT-7 & -\\
NFT-8 & -\\
NFT-9 & -\\
\bottomrule
\end{tabular}
\caption{Trace Between Non-Functional Tests and Modules}
\label{TblNFTM}
\end{table}

\newpage

\subsection{Automated Testing Traceability Matrix}
\begin{table}[H]
\centering
\begin{tabular}{p{0.2\textwidth} p{0.6\textwidth}}
\toprule
\textbf{Test} & \textbf{Modules}\\
\midrule
AT-1 & -\\
AT-2 & -\\
AT-3 & -\\
AT-4 & -\\
AT-5 & -\\
AT-6 & -\\
AT-7 & -\\
AT-8 & -\\
AT-9 & -\\
\bottomrule
\end{tabular}
\caption{Trace Between Automated Tests and Modules}
\label{TblATM}
\end{table}

\newpage
	
\section{Code Coverage Metrics}					% Mario

With the tests mentioned throughout this document, our group has produced an approximate 85\% code coverage. This is evident through the traceability matrices above, as they show that every module has been covered, with some being covered multiple times.

\end{document}
