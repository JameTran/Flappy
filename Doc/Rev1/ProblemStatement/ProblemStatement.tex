\documentclass{article}

\usepackage{booktabs}
\usepackage{tabularx}
\usepackage{hyperref}
\usepackage{float}
\usepackage{graphicx}
\hypersetup{
    colorlinks,
    citecolor=black,
    filecolor=black,
    linkcolor=red,
    urlcolor=blue
}
\usepackage[round]{natbib}
\usepackage{array}
\usepackage{ulem}

\title{SE 3XA3: Problem Statement \\ Mini-Arcade \\{\large L01, Team \#4}}

\author{Andrew Hum \\ huma3 \\ 400138826 \and
		William Lei \\ leim5 \\ 400125240 \and
		Arshan Khan \\ khana172 \\ 400145605 \and
		Jame Tran \\ tranj52 \\ 000000000
}

\date{January 24, 2019}

% \input{../Comments} %Cannot find a comments file in the parent directory

\begin{document}
\maketitle
\newpage

\tableofcontents
\newpage



\begin{table}[hp]
\section{Revision History} \label{TblRevisionHistory}
\begin{tabularx}{\textwidth}{llX}
\hline
\textbf{Date} & \textbf{Developer(s)} & \textbf{Change}\\
\hline
1/21/2020 & Andrew Hum & Initialization \& Rough Ideas.\\
\hline
1/22/2020 & Andrew Hum  & Document Formatting.\\
\hline
1/22/2020 & Arshan Khan & Document Formatting.\\
\hline
1/22/2020 & Andrew Hum & Completed initial draft with a thorough edit. \\
\hline
1/22/2020 & William Lei & Section 2.1 - 2.3.\\
\hline
1/23/2020 & Arshan Khan & Light editing to improve brevity\\
\hline
4/1/2020 & Andrew Hum & Corrected previous feedback for revision 1\\
\hline
\end{tabularx}
\end{table}

\newpage

\section{Problem Statement}
\subsection{What is the problem to be solved?}
The purpose of any game is to entertain the user. However, the term entertain is broad with many interpretations and meanings based on each individual. Our team believes that the core purpose of a good game is to bring joy to the players, leaving them with a desire to play longer. The game should be easily accessible, pose a challenge to the user that provides a feeling of accomplishment when overcome and be aesthetically pleasing in terms of graphics and performance. The current selection of mini-games are simple with minimal functionality, leaving a minimal desire for the user to play. 

\textcolor{red}{The mini-games in their current state do not meet our previously established criteria on user enjoy-ability. Firstly, the games are difficult to install and launch, decreasing the appeal to play them. Alongside the difficulty to access, the mini-games have minimal functionality, low difficulty, and poor overall performance. The current Maze game presents one pre-set maze that can be guided through using the mouse and clicking the desired spot. This game has little-to-no functionality and provides not satisfaction when completed as it is incredibly easy to complete with no victory screen. The current Pong game has incredibly poor performance with visible lag. It also does not keep track of the score which takes away from the satisfaction of scoring a point. The Flappy game is mediocre but has poor graphics quality and doesn't track score, meaning there is no telling if you are currently surpassing a previous high score. Lastly, all these games must be run using the command terminal and opened individually with different commands.}

\subsection{Why is this problem important?}
These previously mentioned aspects cause the interest factor of the player to exhaust rapidly; in other words, the games in their current state are not entertaining and are not enjoyable to play. However, redesigning these mini-games and implementing a launcher for the games has a high potential to improve their enjoyability and create a desirable game as defined above.

\subsection{What is the problems context?}
Mini-Arcade is an application designed for anyone interested, regardless of the individuals' demographic characteristics. With the redesign of these simple games, we anticipate varying difficulty levels for players of different skill levels as well as a variety of mini-games to choose from to satisfy each individual's interests. This application is built for personal computers with very basic hardware requirements: any computer will be able to play the games. As many people have access to a computer, this creates an easily-accessible environment for our project, allowing anyone to utilize Mini-Arcade and have fun.

\textcolor{red}{The current stakeholders are Dr. Asghar Bokhari, the TAs associated with the grading and overview of this project, and the current memebers of the Mini-Arcade development team.}

%- A clear concise description of the issues that need to be addressed by your team
%- What problem are you trying to solve? Not how you are going to solve the problem
%- Why is this an important problem?
%- What is the context of the problem you are solving?
%- Stakeholders?
%- Software environment?
	

%\wss{comment}

%\ds{comment}

%\mj{comment}

%\cm{comment}

%\mh{comment}

\end{document}
