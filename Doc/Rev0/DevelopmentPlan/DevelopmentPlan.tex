\documentclass{article}

\usepackage{booktabs}
\usepackage{tabularx}
\usepackage{hyperref}

\hypersetup{colorlinks=true,urlcolor=cyan}

\title{SE 3XA3: Development Plan\\Mini-Arcade}

\title{SE 3XA3: Problem Statement \\ Mini-Arcade \\{\large Team \#4, Undetermined}}

\author{Andrew Hum \\ huma3 \\ 400138826 \and
		William Lei \\ leim5 \\ 400125240 \and
		Arshan Khan \\ khana172 \\ 400145605 \and
		Jame Tran \\ tranj52 \\ 400144141
}

\date{}

%\input{../Comments}

\begin{document}

\begin{table}[hp]
\caption{Revision History} \label{TblRevisionHistory}
\begin{tabularx}{\textwidth}{llX}
\toprule
\textbf{Date} & \textbf{Developer(s)} & \textbf{Change}\\
\midrule
1/29/2020 & Andrew Hum & Sections 1 - 3\\
1/30/2020 & William Lei & Sections 4, 7\\
1/31/2020 & Arshan Khan & Section 5, 8, 9 \\
1/31/2020 & Jame Tran & Section 5, 6 \\
Date2 & Name(s) & Description of changes\\
... & ... & ...\\
\bottomrule
\end{tabularx}
\end{table}

\newpage

\maketitle

\section{Team Meeting Plan}

Typical meetings will occur during lab time on Mondays and Wednesdays from 9:30 am - 11:20 am in room 236 in the Information Technology Building on campus. These meetings will occur weekly with the exceptions of breaks. During these meetings, Arshan leads the group as the project manager, however, the group works as equals to assign appropriate work to each individual. If there is work left uncompleted after our allocated meeting slots, additional meetings will be scheduled amongst the team members. The individual in charge of the meeting agenda will switch each meeting and be in charge of recording minutes, written statement on decisions made and homework designations.

\section{Team Communication Plan}

The team's primary means of communication is through a group chat on Facebook. This allows easy communication to the entire group, as well as individual conversations if needed. For issues related to the software, we may use Git Issues to notify one another on current issues that need resolving. Each group member can contact one another through email if necessary.

\section{Team Member Roles}

Arshan Khan - Project Manager, Game Developer, Tester \\
Andrew Hum - Architect, Game Developer, Tester \\
William Lei - Git Expert, Launcher Developer \\
Jame Tran - Developer, Tester

\section{Git Workflow Plan}

The git workflow of the team will be centralized and branch-based. There will a total of 6 branches: master; development; and a personal branch for each team member.\\
The master branch is the central and main branch, it will always contain production-ready code. Team members cannot directly commit changes to the master branch, as it can only be updated by merging changes from the development branch. The main branch will be labeled by a version number along with a title, which will be updated after every merge.\\
The development branch is the main branch that will be used for the development of the project. It is initialized by branching off master and can be reset to master when needed. Similar to master branch, team members cannot directly commit to this branch, as it can only be updated by merging from their personal branch.\\
Each team member will have a personal branch that only that person has the permission to commit to. This is the branch that the team members commit their code to during the development.\\
Each member can update the development branch anytime by merging their personal branch if the code in their personal branch is up to date with the changes made in the development branch (by other team members) and fully operational. When a milestone or a certain checkpoint is met, upon the agreement of all team members, the master will then be updated by merging from the development branch, and its label will be updated with a newer version number and a new title.

\section{Proof of Concept Demonstration Plan}

The most significant risk for our game refresh and launcher creation will be that recreating multiple games with refreshed visuals and different levels of difficulty will prove to be a challenge, especially in the given time frame of the project.

\iffalse
Also, we may find testing to be in the middle between manageable and complicated. On one hand, running pytest for some unit tests will be simple but on the other, usability tests, such as if the games and launcher will run pleasingly, will need to be conducted manually and may more time than expected.
\fi

Testing the game may prove to be a challenge. We plan to have at least three games available to play, this means each game as well as the launcher needs to be sufficiently tested.  While small unit tests can be quickly be written using pytest, any system testing will have to be conducted manually by playing the game, or by automating parts of the game, which can be extremely time-consuming. Furthermore, any UI components or concerns such as if the game runs pleasingly will also have to be tested manually.

The required libraries are not too difficult to install, the most challenging aspect will be the actual programming of the games. That said, portability is expected to be very seamless as the project is intended to be downloaded as an executable file online. Once the executable file is downloaded, the game will install itself.

To overcome the above risks, we will spend time slowly building the games in advance. This is shown in the Gantt Chart in Section \ref{gantt}, where we spend nearly two weeks on the proof of concept to have one game and the launcher to be functional. Another risk to overcome is the amount of testing that needs to be done.  We will spend time building test cases as we have learned in our other courses. Making good use of the many features of pytest such as mock patches, as well as following development techniques such as test-driven development will allow us to write quick, efficient unit tests. For system testing, the games being developed are not overly difficult, and can feasibly be beaten by the developers. We will be sure to allocate adequate time to manually play and test the games. The final risk we will face is packaging the game properly to install correctly on different machines. We will test this on multiple computers as well as different operating systems to ensure correct installation.

\section{Technology}

The project will be developed using Python 3. Python was chosen due to a large number of modules available for development, as well as its familiarity with the team developers. In this project, we will be using Pygame, which is a python wrapper for the SDL library, which controls the mouse, keyboard, sound, and video on the user's system. Pygame also adds functionality to the SDL library by providing collision detection, vector math, and MIDI support. 

For IDEs, we will be using the Visual Studio Code text editor (VSCode). VSCode is a powerful, lightweight text editor with extensions that allow real-time linting, code style enforcement and debugging. Any additional tools can be run from the terminal. 

For the testing framework, we will be using pytest. pytest is a lightweight testing framework that provides several useful features, such as parameterized tests.

For document generation, we will be using Doxygen. Doxygen is a documentation generating tool that is run on specially annotated Python code. Doxygen can then create an online document browser using HTML, or an offline reference guide using \LaTeX.


\section{Coding Style}

PEP 8 will be used as the coding style for this project. The detail style guide can be found on the official website of Python at \url{http://www.python.org/dev/peps/pep-0008}.\\
Doxygen will be used as the documentation and commenting style guide for this project. The detail style guide can be found on the official website of Doxygen at \url{http://www.doxygen.nl/}.

\section{Project Schedule} \label{gantt}

The Gantt Chart can be found at this location \href{run:../../ProjectSchedule/3XA3-ProjSched.gan}{../../ProjectSchedule/3XA3-ProjSched.gan}. \\
The Chart will be updated with more precise tasks once future deliverables are made more clear.

\section{Project Review}
Will be completed in Revision 1.

\end{document}