\documentclass[12pt, titlepage]{article}

\usepackage{booktabs}
\usepackage{tabularx}
\usepackage{hyperref}
\hypersetup{
    colorlinks,
    citecolor=black,
    filecolor=black,
    linkcolor=red,
    urlcolor=blue
}
\usepackage[round]{natbib}

\title{SE 3XA3: Test Plan\\Mini-Arcade}

\author{Team \#104
		\\ Andrew Hum, 400138826
		\\ Arshan Khan, 400145605
		\\ Jame Tran, 400144141
		\\ William Lei, 400125240
}
\date{February 28, 2020}

% \input{../Comments}

\begin{document}

\maketitle

\pagenumbering{roman}
\tableofcontents
\listoftables
\listoffigures

\begin{table}[bp]
\caption{\bf Revision History}
\begin{tabularx}{\textwidth}{p{3cm}p{2cm}X}
\toprule {\bf Date} & {\bf Version} & {\bf Notes}\\
\midrule
2/24/2020 & 1.0 & Andrew and Arshan divided the project into workable parts for group members and began the rough draft of sections 2, 5, 6\\
2/26/2020 & 2.0 & Andrew and Arshan completed section 1 and 2\\
Date 2 & 1.1 & Notes\\
\bottomrule
\end{tabularx}
\end{table}

\newpage

\pagenumbering{arabic}

\section{General Information}

\subsection{Purpose}

The purpose for testing our project is to verify that it meets the requirements outlined in the 'Software Requirements Specification' and ensure that it is implemented correctly.

\subsection{Scope}

The test plan develops a baseline for testing the functionality and correctness of Mini-Arcade. It's core objective is to verify that the games run correctly and efficiently all with a single click utilizing the launcher. The test plan documents will highlight what is to be tested of our project, testing methods and what resources we will use to test our software.

\subsection{Acronyms, Abbreviations, and Symbols}
	
\begin{table}[hbp]
\caption{\textbf{Table of Abbreviations}} \label{Abbrev}

\begin{tabularx}{\textwidth}{p{3cm}X}
\toprule
\textbf{Abbreviation} & \textbf{Definition} \\
\midrule
Abbreviation & Definition\\
\bottomrule
\end{tabularx}

\end{table}

\begin{table}[!htbp]
\caption{\textbf{Table of Definitions}} \label{Defs}

\begin{tabularx}{\textwidth}{p{3cm}X}
\toprule
\textbf{Term} & \textbf{Definition}\\
\midrule
Term1 & Definition1\\
\bottomrule
\end{tabularx}

\end{table}	

\subsection{Overview of Document}

This document will outline a detailed testing plan with the tools that will be utilized and the approximated schedule of testing. It will also give in-depth test cases and the method of testing for the functional requirements, non-functional requirements, the proof of concept tests and the unit-testing plan.

\section{Plan}
	
\subsection{Software Description}

The software is a launcher for a selection of games for the user to play. These games are updated from their original versions to be more visually pleasing and challenging.

\subsection{Test Team}

The test team is composed of all team members: Andrew Hum, Arshan Khan, Jame Tran, and William Lei.

\subsection{Automated Testing Approach}

The tests will be automated by pytest because it is very popular and allows "assert rewriting".

\subsection{Testing Tools}

\subsection{Testing Schedule}
		
See Gantt Chart at the following url: \\
\href{run:../../ProjectSchedule/3XA3-ProjSched.gan}{../../ProjectSchedule/3XA3-ProjSched.gan}. 

\section{System Test Description}
	
\subsection{Tests for Functional Requirements}

\subsubsection{Area of Testing1}
		
\paragraph{Title for Test}

\begin{enumerate}

\item{test-id1\\}

Type: Functional, Dynamic, Manual, Static etc.
					
Initial State: 
					
Input: 
					
Output: 
					
How test will be performed: 
					
\item{test-id2\\}

Type: Functional, Dynamic, Manual, Static etc.
					
Initial State: 
					
Input: 
					
Output: 
					
How test will be performed: 

\end{enumerate}

\subsubsection{Area of Testing2}

...

\subsection{Tests for Nonfunctional Requirements}

\subsubsection{Area of Testing1}
		
\paragraph{Title for Test}

\begin{enumerate}

\item{test-id1\\}

Type: 
					
Initial State: 
					
Input/Condition: 
					
Output/Result: 
					
How test will be performed: 
					
\item{test-id2\\}

Type: Functional, Dynamic, Manual, Static etc.
					
Initial State: 
					
Input: 
					
Output: 
					
How test will be performed: 

\end{enumerate}

\subsubsection{Area of Testing2}

...

\subsection{Traceability Between Test Cases and Requirements}

\section{Tests for Proof of Concept}

\subsection{Area of Testing1}
		
\paragraph{Title for Test}

\begin{enumerate}

\item{test-id1\\}

Type: Functional, Dynamic, Manual, Static etc.
					
Initial State: 
					
Input: 
					
Output: 
					
How test will be performed: 
					
\item{test-id2\\}

Type: Functional, Dynamic, Manual, Static etc.
					
Initial State: 
					
Input: 
					
Output: 
					
How test will be performed: 

\end{enumerate}

\subsection{Area of Testing2}

...
	
\section{Unit Testing Plan}

The pytest library will also be used for the unit testing of our project.
		
\subsection{Unit testing of internal functions}

To unit test the project, hard-coded, expected inputs will be given to the individual functions and methods. These functions and methods will then provide output, and we will check that these are the expected output given our scenarios. As games are more difficult to completely test with unit tests, we will only test the functions that can be tested by providing an expected and unexpected output with simple input values. To cover a wide range of scenarios, the input variables will test both expected output, and reaction to incorrect/unexpected input values. There will be no need for stubs or drivers to test our project. To ensure high-quality coverage, we will be using testing coverage metrics. Our goal is to cover a minimum of 60\% of the project with unit tests alone, derived by the total lines of code in the project divided by the number of lines covered by the test cases.
	
\subsection{Unit testing of output files}	

In-depth testing of the output files using unit testing will be not applicable for our project, and any unit tests to test output files would prove to be not useful.

\bibliographystyle{plainnat}

\bibliography{SRS}

\newpage

\section{Appendix}

This is where you can place additional information.

\subsection{Symbolic Parameters}

The definition of the test cases will call for SYMBOLIC\_CONSTANTS.
Their values are defined in this section for easy maintenance.

\subsection{Usability Survey Questions?}

This is a section that would be appropriate for some teams.
A possible set of questions to ask beta testers would include:
\begin{itemize}
    \item What game did you play first?
    \item What did you think of that game?
    \item How long did you play?
    \item Would you play again?
    \item Did you play any other games?
    \item Was it easy to get started?
\end{itemize}

\end{document}
