\documentclass[12pt, titlepage]{article}

\usepackage{booktabs}
\usepackage{tabularx}
\usepackage{hyperref}
\hypersetup{
    colorlinks,
    citecolor=black,
    filecolor=black,
    linkcolor=red,
    urlcolor=blue
}
\usepackage[round]{natbib}

\title{SE 3XA3: Test Plan\\Mini-Arcade}

\author{Team \#104
		\\ Andrew Hum, 400138826
		\\ Arshan Khan, 400145605
		\\ Jame Tran, 400144141
		\\ William Lei, 400125240
}
\date{\today}

% \input{../Comments}

\begin{document}

\maketitle

\pagenumbering{roman}
\tableofcontents
\listoftables
\listoffigures

\begin{table}[bp]
\caption{\bf Revision History}
\begin{tabularx}{\textwidth}{p{3cm}p{2cm}X}
\toprule {\bf Date} & {\bf Version} & {\bf Notes}\\
\midrule
2/24/2020 & 1.0 & Andrew and Arshan divided the project into workable parts for group members and began the rough draft of sections 1, 2, 5\\
2/26/2020 & 1.1 & Andrew completed sections 1 and 5\\
2/27/2020 & 1.2 & Andrew revised sections 1 and 5 for grammatical errors\\
2/27/2020 & 1.3 & William completed section 3.1\\
\bottomrule
\end{tabularx}
\end{table}

\newpage

\pagenumbering{arabic}

\section{General Information}

\subsection{Purpose}

The purpose of testing our project is to verify that it meets the requirements outlined in the 'Software Requirements Specification' and ensure that it is implemented correctly.

\subsection{Scope}

The test plan develops a baseline for testing the functionality and correctness of Mini-Arcade. Its core objective is to verify that the games run correctly and efficiently all with a single click utilizing the launcher. The test plan documents will highlight what is to be tested of our project, testing methods and what resources we will use to test our software.

\subsection{Acronyms, Abbreviations, and Symbols}
	
\begin{table}[hbp]
\caption{\textbf{Table of Abbreviations}} \label{Abbrev}

\begin{tabularx}{\textwidth}{p{3cm}X}
\toprule
\textbf{Abbreviation} & \textbf{Definition} \\
\midrule
FDM & Functional, Dynamic and Manual Testing\\
\bottomrule
\end{tabularx}

\end{table}

\begin{table}[!htbp]
\caption{\textbf{Table of Definitions}} \label{Defs}

\begin{tabularx}{\textwidth}{p{3cm}X}
\toprule
\textbf{Term} & \textbf{Definition}\\
\midrule
Functional Testing & Testing derived from the functional requirements of the software.\\
Dynamic Testing & Testing through executing test cases during runtime.\\
Manual Testing & Testing conducted by providing manual inputs and people checking for outputs.\\
\bottomrule
\end{tabularx}

\end{table}	

\subsection{Overview of Document}

This document will outline a detailed testing plan with the tools that will be utilized and the approximated schedule of testing. It will also give in-depth test cases and the method of testing for the functional requirements, non-functional requirements, the proof of concept tests and the unit-testing plan.

\section{Plan}
	
\subsection{Software Description}

The software is a launcher for a selection of games for the user to play. These games are updated from their original versions to be more visually pleasing and challenging.

\subsection{Test Team}

The test team is composed of all team members: Andrew Hum, Arshan Khan, Jame Tran, and William Lei.

\subsection{Automated Testing Approach}

The tests will be automated by pytest because it is very popular and allows "assert rewriting".

\subsection{Testing Tools}

\subsection{Testing Schedule}
		
See Gantt Chart at the following url: \\
\href{run:../../ProjectSchedule/3XA3-ProjSched.gan}{../../ProjectSchedule/3XA3-ProjSched.gan}. 

\section{System Test Description}
	
\subsection{Tests for Functional Requirements}

\subsubsection{General Navigation}

\begin{enumerate}

\item{FR-N-1\\}
Type: FDM\\
Initial State: Main Screen\\
Input: User clicks on Leaderboard\\
Output: Leaderboard opens and is displayed on the screen.\\
How test will be performed: The application will be opened and the user will manually provides inputs to the software and observes for the output of the software on the screen.\\

\item{FR-N-2\\}
Type: FDM\\
Initial State: Main Screen\\
Input: User clicks on Maze\\
Output: The mini-game Maze opens and is displayed on the screen.\\
How test will be performed: The application will be opened and the user will manually provides inputs to the software and observes for the output of the software on the screen.\\

\item{FR-N-3\\}
Type: FDM\\
Initial State: Main Screen\\
Input: User clicks on Flappy\\
Output: The mini-game Flappy opens and is displayed on the screen.\\
How test will be performed: The application will be opened and the user will manually provides inputs to the software and observes for the output of the software on the screen.\\

\item{FR-N-4\\}
Type: FDM\\
Initial State: Main Screen\\
Input: User clicks on Pong\\
Output: The mini-game Pong opens and is displayed on the screen.\\
How test will be performed: The application will be opened and the user will manually provides inputs to the software and observes for the output of the software on the screen.\\

\item{FR-N-5\\}
Type: FDM\\
Initial State: Main Screen\\
Input: User clicks on close button\\
Output: The software will be terminated.\\
How test will be performed: The application will be opened and the user will manually provides inputs to the software and observes for the output of the software on the screen.\\

\item{FR-N-6\\}
Type: FDM\\
Initial State: Leaderboard Screen\\
Input: User clicks on Maze\\
Output: The leaderboard screen will display the leaderboard for Maze.\\
How test will be performed: The application will be opened and the user will manually provides inputs to the software and observes for the output of the software on the screen.\\

\item{FR-N-7\\}
Type: FDM\\
Initial State: Leaderboard Screen\\
Input: User clicks on Flappy\\
Output: The leaderboard screen will display the leaderboard for Flappy.\\
How test will be performed: The application will be opened and the user will manually provides inputs to the software and observes for the output of the software on the screen.\\

\item{FR-N-8\\}
Type: FDM\\
Initial State: Maze - Menu Screen\\
Input: User clicks on Help\\
Output: The screen will display the instructions for how to play the mini-game.\\
How test will be performed: The application will be opened and the user will manually provides inputs to the software and observes for the output of the software on the screen.\\

\item{FR-N-9\\}
Type: FDM\\
Initial State: Flappy - Menu Screen\\
Input: User clicks on Help\\
Output: The screen will display the instructions for how to play the mini-game.\\
How test will be performed: The application will be opened and the user will manually provides inputs to the software and observes for the output of the software on the screen.\\

\item{FR-N-10\\}
Type: FDM\\
Initial State: Pong - Menu Screen\\
Input: User clicks on Help\\
Output: The screen will display the instructions for how to play the mini-game.\\
How test will be performed: The application will be opened and the user will manually provides inputs to the software and observes for the output of the software on the screen.\\

\item{FR-N-11\\}
Type: FDM\\
Initial State: Maze - Menu Screen\\
Input: User clicks on Leaderboard\\
Output: The leaderboard of the mini-game opens and is displayed on the screen.\\
How test will be performed: The application will be opened and the user will manually provides inputs to the software and observes for the output of the software on the screen.\\

\item{FR-N-12\\}
Type: FDM\\
Initial State: Flappy - Menu Screen\\
Input: User clicks on Leaderboard\\
Output: The leaderboard of the mini-game opens and is displayed on the screen.\\
How test will be performed: The application will be opened and the user will manually provides inputs to the software and observes for the output of the software on the screen.\\

\item{FR-N-13\\}
Type: FDM\\
Initial State: Maze - Menu Screen\\
Input: User clicks on Back\\
Output: The Main Screen opens and is displayed on the screen.\\
How test will be performed: The application will be opened and the user will manually provides inputs to the software and observes for the output of the software on the screen.\\

\item{FR-N-14\\}
Type: FDM\\
Initial State: Flappy - Menu Screen\\
Input: User clicks on Back\\
Output: The Main Screen opens and is displayed on the screen.\\
How test will be performed: The application will be opened and the user will manually provides inputs to the software and observes for the output of the software on the screen.\\

\item{FR-N-15\\}
Type: FDM\\
Initial State: Pong - Menu Screen\\
Input: User clicks on Back\\
Output: The Main Screen opens and is displayed on the screen.\\
How test will be performed: The application will be opened and the user will manually provides inputs to the software and observes for the output of the software on the screen.\\

\end{enumerate}

\subsubsection{Mini-Game - Maze}

\begin{enumerate}

\item{FR-MGM-1\\}
Type: FDM\\
Initial State: Maze - Menu Screen\\
Input: User clicks on a difficulty level\\
Output: A maze will displayed on the screen.\\
How test will be performed: The application will be opened and the user will manually provides inputs to the software and observes for the output of the software on the screen.\\

\item{FR-MGM-2\\}
Type: FDM\\
Initial State: Maze - Game Screen\\
Input: User clicks on home\\
Output: Menu screen of Maze will displayed on the screen.\\
How test will be performed: The application will be opened and the user will manually provides inputs to the software and observes for the output of the software on the screen.\\

\item{FR-MGM-3\\}
Type: FDM\\
Initial State: Maze - Menu Screen\\
Input: User clicks on a specific difficulty level, then clicks home, and repeats this for 5 times in total\\
Output: A maze will displayed on the screen every time the user clicks a difficulty level, and there should be no patterns for when a specific maze will be displayed.\\
How test will be performed: The application will be opened and the user will manually provides inputs to the software and observes for the output of the software on the screen.\\

\item{FR-MGM-4\\}
Type: FDM\\
Initial State: Maze - Game Screen\\
Input: User clicks a movement key\\
Output: The object will move according to the key-movement mapping and the movement will be displayed on the screen.\\
How test will be performed: The application will be opened and the user will manually provides inputs to the software and observes for the output of the software on the screen.\\

\item{FR-MGM-5\\}
Type: FDM\\
Initial State: Maze - Game Screen\\
Input: Object reaches end of maze through a movement\\
Output: A score (base on time elapsed) along with high score will be displayed on the end game screen.\\
How test will be performed: The application will be opened and the user will manually provides inputs to the software and observes for the output of the software on the screen.\\

\item{FR-MGM-6\\}
Type: FDM\\
Initial State: Maze - End Game Screen\\
Input: User clicks on Next\\
Output: A maze will displayed on the screen.\\
How test will be performed: The application will be opened and the user will manually provides inputs to the software and observes for the output of the software on the screen.\\

\item{FR-MGM-7\\}
Type: FDM\\
Initial State: Maze - End Game Screen\\
Input: User clicks on Return\\
Output: The Menu Screen opens and is displayed on the screen.\\
How test will be performed: The application will be opened and the user will manually provides inputs to the software and observes for the output of the software on the screen.\\
    
\end{enumerate}

\subsubsection{Mini-Game - Flappy}

\begin{enumerate}

\item{FR-MGF-1\\}
Type: FDM\\
Initial State: Flappy - Menu Screen\\
Input: User clicks on start\\
Output: The game will be initialized/started and the game screen will be opened and displayed\\
How test will be performed: The application will be opened and the user will manually provides inputs to the software and observes for the output of the software on the screen.\\

\item{FR-MGF-2\\}
Type: FDM\\
Initial State: Flappy - Game Screen\\
Input: User controlling the character to make sure it will not collide with any object\\
Output: Their will be randomly generated objects approaching toward the character, and their speed and amount generated will be increased as time elapses.\\
How test will be performed: The application will be opened and the user will manually provides inputs to the software and observes for the output of the software on the screen.\\

\item{FR-MGF-3\\}
Type: FDM\\
Initial State: Flappy - Game Screen\\
Input: User clicks space key for 5 times separated by a short period of time\\
Output: The character will move up a constant amount every time the space key is being clicked.\\
How test will be performed: The application will be opened and the user will manually provides inputs to the software and observes for the output of the software on the screen.\\

\item{FR-MGF-4\\}
Type: FDM\\
Initial State: Flappy - Game Screen\\
Input: User controls the character to collide with an object\\
Output: A score (base on time elapsed) along with high score will be displayed on the end game screen.\\
How test will be performed: The application will be opened and the user will manually provides inputs to the software and observes for the output of the software on the screen.\\

\item{FR-MGF-5\\}
Type: FDM\\
Initial State: Flappy - End Game Screen\\
Input: User clicks on Restart\\
Output: The game will be initialized/started and the game screen will be opened and displayed.\\
How test will be performed: The application will be opened and the user will manually provides inputs to the software and observes for the output of the software on the screen.\\

\item{FR-MGF-6\\}
Type: FDM\\
Initial State: Flappy - End Game Screen\\
Input: User clicks on Return\\
Output: The Menu Screen opens and is displayed on the screen.\\
How test will be performed: The application will be opened and the user will manually provides inputs to the software and observes for the output of the software on the screen.\\
    
\end{enumerate}

\subsubsection{Mini-Game - Pong}

\begin{enumerate}

\item{FR-MGP-1\\}
Type: FDM\\
Initial State: Pong - Menu Screen\\
Input: User clicks on Single Player\\
Output: The game screen will be opened and displayed and will request the user to input a max score.\\
How test will be performed: The application will be opened and the user will manually provides inputs to the software and observes for the output of the software on the screen.\\

\item{FR-MGP-2\\}
Type: FDM\\
Initial State: Pong - Menu Screen\\
Input: User clicks on Multiplayer\\
Output: The game screen will be opened and displayed and will request the user to input a max score.\\
How test will be performed: The application will be opened and the user will manually provides inputs to the software and observes for the output of the software on the screen.\\

\item{FR-MGF-3\\}
Type: FDM\\
Initial State: Pong - Game Screen (Both single and multiplayer, requesting max score input)\\
Input: User inputs a integer between 1 to 10.\\
Output: The game will be initialized or started.
How test will be performed: The application will be opened and the user will manually provides inputs to the software and observes for the output of the software on the screen.\\

\item{FR-MGP-4\\}
Type: FDM\\
Initial State: Pong - Game Screen (Both single and multiplayer)\\
Input: User clicks a movement key\\
Output: The corresponding paddle will move according to the key-movement mapping and the movement will be displayed on the screen.\\
How test will be performed: The application will be opened and the user will manually provides inputs to the software and observes for the output of the software on the screen.\\

\item{FR-MGP-5\\}
Type: FDM\\
Initial State: Pong - Game Screen (Both single and multiplayer)\\
Input: User control the paddle to hit the ball until the ball reaches the boundary on either side (and did not hit a paddle)\\
Output: The score of the opposite will be increased by 1 and the change will be displayed on the game screen.\\
How test will be performed: The application will be opened and the user will manually provides inputs to the software and observes for the output of the software on the screen.\\

\item{FR-MGF-6\\}
Type: FDM\\
Initial State: Flappy - Game Screen (Both single and multiplayer)\\
Input: User control the paddle to hit the ball until either side reaches the max score\\
Output: The score between the two player will be displayed on the end game screen.\\
How test will be performed: The application will be opened and the user will manually provides inputs to the software and observes for the output of the software on the screen.\\

\item{FR-MGP-7\\}
Type: FDM\\
Initial State: Pong - End Game Screen\\
Input: User clicks on Restart\\
Output: The game will be initialized/started and the game screen will be opened and displayed.\\
How test will be performed: The application will be opened and the user will manually provides inputs to the software and observes for the output of the software on the screen.\\

\item{FR-MGP-8\\}
Type: FDM\\
Initial State: Pong - End Game Screen\\
Input: User clicks on Return\\
Output: The Menu Screen opens and is displayed on the screen.\\
How test will be performed: The application will be opened and the user will manually provides inputs to the software and observes for the output of the software on the screen.\\
    
\end{enumerate}

\subsection{Tests for Nonfunctional Requirements}

\subsubsection{Area of Testing1}
		
\paragraph{Title for Test}

\begin{enumerate}

\item{test-id1\\}

Type: 
					
Initial State: 
					
Input/Condition: 
					
Output/Result: 
					
How test will be performed: 
					
\item{test-id2\\}

Type: Functional, Dynamic, Manual, Static etc.
					
Initial State: 
					
Input: 
					
Output: 
					
How test will be performed: 

\end{enumerate}

\subsubsection{Area of Testing2}

...

\subsection{Traceability Between Test Cases and Requirements}

\section{Tests for Proof of Concept}

\subsection{Area of Testing1}
		
\paragraph{Title for Test}

\begin{enumerate}

\item{test-id1\\}

Type: Functional, Dynamic, Manual, Static etc.
					
Initial State: 
					
Input: 
					
Output: 
					
How test will be performed: 
					
\item{test-id2\\}

Type: Functional, Dynamic, Manual, Static etc.
					
Initial State: 
					
Input: 
					
Output: 
					
How test will be performed: 

\end{enumerate}

\subsection{Area of Testing2}

...
	
\section{Unit Testing Plan}

The pytest library will be used for the unit testing of our project.
		
\subsection{Unit testing of internal functions}

To efficiently use unit-testing for our project, we will use hard-coded, expected, and unexpected, inputs for individual functions and methods. These functions and methods will then provide output, and we will verify that the resulting output is correct or that the program handles the unexpected input correctly. For example, telling the game that the game was won, and the expected output should be the end-game screen. As games are more difficult to completely test with unit tests, we will only test the functions that can be tested by providing an expected and unexpected output with input values relating to a current state or completed event. To cover a wide range of scenarios, the input variables will test both expected output, and reaction to incorrect/unexpected input values. There will be no need for stubs or drivers to test our project. To ensure high-quality coverage, we will be using testing coverage metrics. Our goal is to cover a minimum of 60\% of the project with unit tests alone, derived by the total lines of code in the project divided by the number of lines covered by the test cases.
	
\subsection{Unit testing of output files}	

In-depth testing of the output files using unit testing will be not applicable for our project, and any unit tests to test output files would prove to be not useful and ineffective in both coverage and effective use of time.

\bibliographystyle{plainnat}

\bibliography{SRS}

\newpage

\section{Appendix}

This is where you can place additional information.

\subsection{Symbolic Parameters}

The definition of the test cases will call for SYMBOLIC\_CONSTANTS.
Their values are defined in this section for easy maintenance.

\subsection{Usability Survey Questions?}

This is a section that would be appropriate for some teams.
A possible set of questions to ask beta testers would include:
\begin{itemize}
    \item What game did you play first?
    \item What did you think of that game?
    \item How long did you play?
    \item Would you play again?
    \item Did you play any other games?
    \item Was it easy to get started?
\end{itemize}

\end{document}
