\documentclass[12pt, titlepage]{article}

\usepackage{fullpage}
\usepackage[round]{natbib}
\usepackage{multirow}
\usepackage{booktabs}
\usepackage{tabularx}
\usepackage{graphicx}
\usepackage{float}
\usepackage{hyperref}
\hypersetup{
    colorlinks,
    citecolor=black,
    filecolor=black,
    linkcolor=red,
    urlcolor=blue
}
\usepackage[round]{natbib}

\newcounter{acnum}
\newcommand{\actheacnum}{AC\theacnum}
\newcommand{\acref}[1]{AC\ref{#1}}

\newcounter{ucnum}
\newcommand{\uctheucnum}{UC\theucnum}
\newcommand{\uref}[1]{UC\ref{#1}}

\newcounter{mnum}
\newcommand{\mthemnum}{M\themnum}
\newcommand{\mref}[1]{M\ref{#1}}

\title{SE 3XA3: Module Interface Specification\\Mini-Arcade}

\author{Andrew Hum \\ huma3 \\ 400138826 \and
		William Lei \\ leim5 \\ 400125240 \and
		Arshan Khan \\ khana172 \\ 400145605 \and
		Jame Tran \\ tranj52 \\ 400144141
}

\date{}

%\input{../Comments}

\begin{document}

\newpage

\maketitle
\maketitle
\newpage
\tableofcontents

\begin{table}[bp]
\caption{\bf Revision History}
\begin{tabularx}{\textwidth}{p{3cm}p{2cm}X}
\toprule {\bf Date} & {\bf Version} & {\bf Notes}\\
\midrule
3/9/2020 & 1.0 & Arshan and Andrew created document and sections\\
3/11/2020 & 1.1 & Andrew completed all modules relevant to Maze\\
\bottomrule
\end{tabularx}
\end{table}

\newpage

\begin{table}[h!]
    \centering
    \begin{tabular}{p{0.3\textwidth} p{0.6\textwidth}}
    \toprule
    \textbf{Level 1} & \textbf{Level 2}\\
    \midrule
        {Hardware-Hiding Module} & ~ \\
    \midrule
        \multirow{1}{0.3\textwidth}{Behaviour-Hiding Module} & ?\\
        & \textit{Launcher Modules}\\
        & Draw Game (Maze)\\
        & Player Movement (Maze)\\
        & Menu and Settings (Maze)\\ 
        & Draw Game (Pong)\\
        & Player Movement (Pong)\\
        & Menu and Settings (Pong)\\ 
        & \textit{Flappy Modules}\\
    \midrule
        \multirow{1}{0.3\textwidth}{Software Decision Module} & {?}\\
        & \textit{Launcher Modules}\\
        & Maze Generator (Maze)\\
        & Score Tracking (Maze)\\ 
        & Ball Trajectory (Pong)\\
        & Score Tracking (Pong) \\ 
        & \textit{Flappy Modules}\\
    \bottomrule
\end{tabular}
\caption{Module Hierarchy}
\label{TblMH}
\end{table}

\section{MIS of Launcher Module}
		\subsection{Interface Syntax}
		\subsubsection{Exported Access Programs}
		\begin{tabular}[pos]{|c|c|c|c|}
		\end{tabular}
		
		\subsection{Interface Semantics}
		\subsubsection{State Variables}
		
		\subsubsection{Environmental Variables}
		
		\subsubsection{Assumptions}
		
		\subsubsection{Access Program Semantics}
	
	
	
\section{MIS of Maze Generation Module}
		\subsection{Interface Syntax}
		\subsubsection{Exported Access Programs}
		\begin{tabular}[pos]{|c|c|c|c|}
			\hline
			\textbf{Name}& \textbf{In} & \textbf{Out} & \textbf{Exceptions} \\ \hline
			Cell & integer, integer & - & Invalid Input \\ \hline
			Cell\_genWalls & - & - & - \\ \hline
			Maze & integer & - & Invalid Input \\ \hline
			Maze\_genMaze & - & - & - \\ \hline
		\end{tabular}
		
		\subsection{Interface Semantics}
		\subsubsection{State Variables}
		cellWalls: array of integers - representing walls around the current cell
		mazeWalls: array of Cell - representing the layout of the maze
		
		\subsubsection{Assumptions}
		Variables should be set before trying to access them \\ 
		Constructor Cell will be called before genWalls or Maze can be called \\
	    Constructor Maze will be called before genMaze can be called \\
		
		\subsubsection{Access Program Semantics}
		Cell(id, gridLength):
		
		Input: two integers representing the cell ID and the maze dimensions
		
		Transition: initializes the Cell object
		
		Exceptions: Invalid Input they are not positive integers\\
		\\
		Cell\_genWalls():
		
		Input: None
		
		Transition: adds integers corresponding to neighbouring cells to cellWalls
		
		Exceptions: None\\
		\\
		Maze(size):
		
		Input: integer representing the size of the maze
		
		Transition: initializes the Maze Object
		
		Exceptions: Invalid Input size is not a positive integer\\
		\\
		Maze\_genMaze():
		
		Input: None
		
		Transition: utilizes Prim's Algorithm to randomly remove walls from the maze 
		
		and manipulates mazeWalls to represent the remaining walls of the maze
		
		Exceptions: None\\
		\\

\section{MIS of Score Tracking (Maze) Module}
		\subsection{Interface Syntax}
		\subsubsection{Exported Access Programs}
		\begin{tabular}[pos]{|c|c|c|c|}
			\hline
			\textbf{Name}& \textbf{In} & \textbf{Out} & \textbf{Exceptions} \\ \hline
			saveScore & float & - & Invalid Input \\ \hline
			checkRank & float & integer & Invalid Input \\ \hline
		\end{tabular}
		
		\subsection{Interface Semantics}
		\subsubsection{State Variables}
		score: float - represents the user's score once the maze is completed
		\subsubsection{Assumptions}
		Variables should be set before trying to access them
		
		\subsubsection{Access Program Semantics}
		saveScore(time):
		
		Input: a float value representing the total elapsed time during the game
		
		Transition: saves the score to the maze scores file
		
		Exceptions: Invalid Input if the input is not a positive float \\
		\\
		checkRank(time):
		
		Output: the user's current rank based upon previous scores
		
		Exceptions: Invalid Input if the input is not a positive float \\

\section{MIS of Draw Game (Maze) Module}
		\subsection{Interface Syntax}
		\subsubsection{Exported Access Programs}
		\begin{tabular}[pos]{|c|c|c|c|}
			\hline
			\textbf{Name}& \textbf{In} & \textbf{Out} & \textbf{Exceptions} \\ \hline
			drawMaze & Maze & GUI & Invalid Input \\ \hline
			drawCharacter & integer, integer & GUI & Invalid Input \\ \hline
			showTime & - & GUI & - \\ \hline
		\end{tabular}
		
		\subsection{Interface Semantics}
		\subsubsection{State Variables}
		charPos: x,y - coordinates of the character's current position\\
		timeElapsed: float - represents the current time elapsed
		\subsubsection{State Variables}
		keyDown: captures which key is currently being pressed down
		\subsubsection{Assumptions}
		Variables should be set before trying to access them \\ 
		Maze must be properly initialized before drawTime can be called \\
		
		\subsubsection{Access Program Semantics}
		
		drawMaze(Maze):
		
		Input: Maze object used to draw the layout
		
		Output: draws the maze to the output window
		
		Exceptions: Invalid Input if the object is not of type Maze\\
		\\
		drawCharacter(startx,starty):
		
		Input: two integers representing the coordinates to draw the character
		
		Transition: adjusts charPos based on keyDown using Player Movement Module
		
		Output: character is drawn according to it's current position of the maze
		
		Exceptions: Invalid input if the integers are not of the correct coordinates\\
		\\
		showTime(time):
		
		Input: a float representing the current time elapsed
		
		Output: a clock on the output window representing the current time elapsed
		
		Exceptions: Invalid Input if the input is not a float or negative\\
		\\
		
\section{MIS of Player Movement (Maze) Module}
		\subsection{Interface Syntax}
		\subsubsection{Exported Access Programs}
		\begin{tabular}[pos]{|c|c|c|c|}
			\hline
			\textbf{Name}& \textbf{In} & \textbf{Out} & \textbf{Exceptions} \\ \hline
			moveUp & - & integer, integer & - \\ \hline
			moveDown & - & integer, integer & - \\ \hline
			moveLeft & - & integer, integer & - \\ \hline
			moveRight & - & integer, integer & - \\ \hline
		\end{tabular}
		
		\subsection{Interface Semantics}
		\subsubsection{State Variables}
		charPos: int, int - representing the character's current position as coordinates (x,y)
		\subsubsection{Environment Variables}
		None
		\subsubsection{Assumptions}
		Variables should be set before trying to access them
		
		\subsubsection{Access Program Semantics}
		moveUp():
		
		Input: None
		
		Transition: Adjust charPos upwards (decrease y coordinate) 
		
		Output: two integers representing the new position of the character
		
		Exceptions: None\\
		\\
		moveDown():
		
		Input: None
		
		Transition: Adjust charPos downwards (increase y coordinate)
		
		Output: two integers representing the new position of the character
		
		Exceptions: None\\
		\\
		moveLeft():
		
		Input: None
		
		Transition: Adjust charPos to the left (decrease x coordinate)
		
		Output: two integers representing the new position of the character
		
		Exceptions: None\\
		\\
		moveRight():
		
		Input: None
		
		Transition: Adjust charPos to the right (increase x coordinate) 
		
		Output: two integers representing the new position of the character
		
		Exceptions: None\\
		\\
		
\section{MIS of Menu and Settings (Maze) Module}
		\subsection{Interface Syntax}
		\subsubsection{Exported Access Programs}
		\begin{tabular}[pos]{|c|c|c|c|}
			\hline
			\textbf{Name}& \textbf{In} & \textbf{Out} & \textbf{Exceptions} \\ \hline
			drawInterface & integer & GUI & - \\ \hline
			checkEvent & float, float, boolean & integer & Invalid Input \\ \hline
			
		\end{tabular}
		
		\subsection{Interface Semantics}
		\subsubsection{State Variables}
		currState: int - represents the game's current state
		
		\subsubsection{Environment Variables}
		mousePos: the mouse/pointer's current position
		mouseEvent: captures a mouse event 
		\subsubsection{Assumptions}
		Variables should be set before trying to access them \\ 
		If no event is chosen, checkEvent returns a default value 0 \\
		If currState is 0, drawInterface does not change \\
		
		\subsubsection{Access Program Semantics}
		drawInterface(currState):
		
		Input: an integer representing the game's current state
		
		Output: draws the interface corresponding to the current state to the output window
		
		Exceptions: Invalid Input if the input doesn't correspond to a game state\\
		\\
		checkEvent(xpos, ypos, clicked):
		
		Input: float values representing the mouse's current position on the screen and if the mouse has been clicked
		
		Transition: determines if the current position represents a specified event
		
		Output: integer representing the current state of the game based on the mouse
		
		Exceptions: Invalid Input if the coordinates are not part of the window\\
		\\
		
	
\section{MIS of Pong Module}
	\subsection{Interface Syntax}
		\subsubsection{Exported Access Programs}
		\begin{table}[!htbp]
		\begin{tabular}{|c|c|c|c|}
			\hline
			Name & In & Out & Exceptions \\ \hline

		\end{tabular}
	\end{table}
		
	\subsection{Interface Semantics}
		\subsubsection{State Variables}
		\subsubsection{Environmental Variables}
		\subsubsection{Assumptions}
		\subsubsection{Access Program Semantics}


\section{MIS of Flappy Module}
	\subsection{Interface Syntax}
		\subsubsection{Exported Access Programs}
		
	\begin{tabular}[pos]{|c|c|c|c|}
	\hline
	\textbf{Name}& \textbf{In} & \textbf{Out} & \textbf{Exceptions} \\ 
	\hline
					
	\end{tabular}		
		
	\subsection{Interface Semantics}
		\subsubsection{State Variables}
		
		\subsubsection{Environmental Variables}
		\subsubsection{Assumptions}
		\subsubsection{Access Program Semantics}
		
\end{document}
